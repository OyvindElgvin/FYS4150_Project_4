\documentclass[12pt,english,a4paper]{article}
\usepackage[utf8]{inputenc}
\usepackage[T1]{fontenc}
\usepackage{babel,amsmath,amsthm,graphicx,mathtools,textcomp,varioref,amssymb,float,listings}
\usepackage[top=50pt,left=40pt,right=40pt]{geometry}
\usepackage{titling,wrapfig}
\usepackage{color}
 
\definecolor{codegreen}{rgb}{0,0.6,0}
\definecolor{codegray}{rgb}{0.5,0.5,0.5}
\definecolor{codepurple}{rgb}{0.58,0,0.82}
\definecolor{backcolour}{rgb}{0.95,0.95,0.92}
 
\lstdefinestyle{mystyle}{
    backgroundcolor=\color{backcolour},   
    commentstyle=\color{codegreen},
    keywordstyle=\color{magenta},
    numberstyle=\tiny\color{codegray},
    stringstyle=\color{codepurple},
    basicstyle=\footnotesize,
    breakatwhitespace=false,         
    breaklines=true,                 
    captionpos=b,                    
    keepspaces=true,                 
    numbers=left,                    
    numbersep=5pt,                  
    showspaces=false,                
    showstringspaces=false,
    showtabs=false,                  
    tabsize=2
}
 
\lstset{style=mystyle}



\let\vecarrow\vec
\renewcommand{\vec}[1]{\mathbf{#1}}
\let\oldhat\hat
\renewcommand{\hat}[1]{\oldhat{\mathbf{#1}}}
\def\doubleunderline#1{\underline{\underline{#1}}}
\linespread{1.6}
\renewcommand{\qedsymbol}{$\blacksquare$}

\let\~\tilde

\newcommand{\justdiff}[1]{\frac{\partial}{\partial#1}}
\newcommand{\pdiff}[2]{\frac{\partial #1}{\partial#2}}
\newcommand{\ppdiff}[2]{\frac{\partial^2 #1}{\partial#2^2}}
\newcommand{\proofsquare}{\begin{proof}[] \end{proof}}
\let\f\frac
\DeclareMathOperator{\arccosh}{arcosh}

\DeclarePairedDelimiter\abs{\lvert}{\rvert}%
\DeclarePairedDelimiter\norm{\lVert}{\rVert}%


\tolerance = 5000
\hbadness = \tolerance
\pretolerance = 2000

\title{FYS4150: Project 3}
\author{Jonathan Brakstad Waters\\Øyvind Engebretsen Elgvin\\Henrik Lind Petlund}

\begin{document}

\begin{titlepage}
\maketitle
\begin{abstract}
    
\end{abstract}

\end{titlepage}

\section{Introduction}

The Ising model is a model that has a large variety of usage. In this report, the model will be used to study phase transitions in magnetic systems. The systems at hand are systems of size $L\times L$ particles in a grid. Each particle may be in either state "up" or "down" denoted with $1$ and $-1$, respectively. The model only focuses on the interaction between the nearest neighbors particles, and the energy between these is given by
$$
E=-J\sum_{<kl>}^{N}s_ks_l \label{eq:1}
$$

where $N$ is the number of particles, $s_i=\pm 1$ and $J>0$ is a constant which represents ferromagnetism and the strength of the particle interactions. We assume periodic boundary conditions so that every particle has four nearest neighbors. 

Further, the partition function of the system is given by the energy as

$$
Z=\sum_i e^{-\beta E_i} = \sum_E \Omega (E) e^{-\beta E} \label{eq:2}
$$


\vfill

\section{Methods}

Her tror jeg det må være en slags intro til metodene, hva tror dere? 
Eller at bare hele denne bolken skal i resultater? Det er liksom så mye utregninger. 

\vfill

For the $2x2$ case, the energy, magnetic momentum, specific heat capacity, and the magnetic susceptibility can all be calculated analytically. These analytical values are used as benchmarks for the numerically derived values. 

Finding the analytical values, we start by calculating the energies for the configurations of all possible spins. The number of configurations in the $2x2$ case is  $2^4 = 16$, in total. The energies are given by summing the interaction between the four neighbors:
\begin{align*}
    E = -J \sum_{<kl>}^N s_ks_l 
\end{align*}
where an interaction is one of the four possible:
\begin{align*}
    E_{\uparrow \uparrow} = E_{\downarrow \downarrow} = -2J, 
\end{align*}
\begin{align*}
    E_{\uparrow \downarrow} = E_{\downarrow \uparrow} = 2J
\end{align*}
The associated magnetic moment, $M$, is calculated for each spin configuration as:
\begin{align*}
    M_{i}=\sum_{j=1}^{N} s_{j}
\end{align*}
and completes table 1, displaying the energies, degeneracies, and magnetization for the sixteen spin configurations. The calculations of the energies are shown in the appendix. 

With these energies, degeneracies, magnetization, and periodic boundary conditions we can calculate the analytical benchmark expressions, starting with the participation function, which is calculated as: 
\begin{align*}
    Z = \sum_{i=1}^{2^n} e^{-\beta E_i}
      = 2e^{\beta 8J} + 2e^{-\beta 8J} + 12
      = 4cosh(\beta 8J) + 12
\end{align*}
The expectation value for the energy:
\begin{align*}
    <E> \hspace{0.2cm} 
        = \sum_{i=1}^{2^n} E_iP_i(T) 
        = \frac{1}{Z} \sum_{i=1}^{16} E_ie^{-\beta E_i}
        = - \frac{\delta ln(Z(T))}{\delta \beta}
        = -\frac{32sinh(\beta 8J)}{4cosh(\beta 8J) + 12}
\end{align*}
and the expression for the expectation value for the energy squared:
\begin{align*}
    <E^2> \hspace{0.2cm} 
          = \sum_{i=1}^{2^n} E_i^2P_i(T) 
          = \frac{1}{Z} \sum_{i=1}^{16} E_i^2e^{-\beta E_i}
          = \frac{128J^2e^{\beta 8J} + 128J^2e^{-\beta 8J}}
                 {4cosh(\beta 8J) + 12}
          = \frac{64J^2cosh(\beta 8J)}{cosh(\beta 8J) + 3}
\end{align*}
The specific heat capacity take the expression:
\begin{align*}
    C_v = \frac{\sigma_E^2}{k_BT^2} 
\end{align*}
where the variance for the energies, $\sigma_E^2$, is
\begin{align*}
    \sigma_E^2 = \hspace{0.2cm} <E^2> - <E>^2 \hspace{0.2cm}
               = \frac{64J^2cosh(\beta 8J)}{cosh(\beta 8J) + 3}
               - \bigg(-\frac{32sinh(\beta 8J)}{4cosh(\beta 8J) + 12}\bigg)^2
\end{align*}
Similarly, the analytic expressions of the magnetic moment are calculated as:
\begin{align*}
    <M> \hspace{0.2cm} 
        = \sum_{i=1}^{2^n} M_iP_i(\beta) 
        = \frac{1}{Z} \sum_{i=1}^{16} M_ie^{-\beta E_i} = 0 
\end{align*}
and
\begin{align*}
    <M^2> \hspace{0.2cm} 
          = \sum_{i=1}^{2^n} M_i^2P_i(\beta) 
          = \frac{1}{Z} \sum_{i=1}^{16} M_i^2e^{-\beta E_i}
          = \frac{32e^{\beta 8J} + 32}
                 {4cosh(\beta 8J) + 12}
          = \frac{8e^{\beta 8J} + 8}{cosh(\beta 8J) + 3}
\end{align*}
so that the analytical value of the susceptibility is:
\begin{align*}
    \chi = \frac{\sigma_M^2}{k_BT}
\end{align*}
where the magnetic variance is defined as:
\begin{align*}
    \sigma_M^2 = \hspace{0.2cm} <M^2> - <M>^2 \hspace{0.2cm}
               = \hspace{0.2cm} <M^2> - 0
               = \frac{8e^{\beta 8J} + 8}{cosh(\beta 8J) + 3}
\end{align*}

The mean absolute value of the magnetic moment is of interest because it can be a better way to describe the mean magnetization and is given by:
\begin{align*}
    <|M|> \hspace{0.2cm} 
          = \sum_{i=1}^{2^n} |M_i^2P_i(\beta)| 
          = \frac{1}{Z} \sum_{i=1}^{16} |M_i^2e^{-\beta E_i}|
          = \frac{8e^{\beta 8J} + 16}{4cosh(\beta 8J) + 12}
          = \frac{2e^{\beta 8J} + 4}{cosh(\beta 8J) + 3}
\end{align*}

Kan også sette inn den nye $\chi$ med $|M|$, men ja, jeg vet ikke.




\vfill


\begin{table}[h]
    \centering
    \lstinputlisting[firstline=1]{E_M_Configurations.txt}
    %\includegraphics{}
    \caption{\textit{List of configurations of energies and magnetizations}}
    \label{tab:E_M_Configurations}
\end{table}



\section{Results and Discussion}

\section{Conclusion}

\section{Appendix}

Calculations for table 1 for the 2x2 case:
\begin{align*}
    E_{\text{4 spin}\uparrow} = (-2J) + (-2J) + (-2J) + (-2J) = -8J \text{ with degeneracy = 1}
\end{align*}
\begin{align*}
    E_{\text{3 spin}\uparrow} = 2J + (-2J) + 2J + (-2J) = 0 \text{ with degeneracy = 4}
\end{align*}
\begin{align*}
    E_{\text{2 spin}\uparrow} = (-2J) + 2J + (-2J) + 2J = 0 \text{ with degeneracy = 4}
\end{align*}
\begin{align*}
    E_{\text{2 spin}\uparrow} = 2J + 2J + 2J + 2J = 8J \text{ with degeneracy = 2}
\end{align*}
\begin{align*}
    E_{\text{1 spin}\uparrow} = 2J + 2J + (-2J) + (-2J) = 0 \text{ with degeneracy = 4}
\end{align*}
\begin{align*}
    E_{\text{0 spin}\uparrow} = (-2J) + (-2J) + (-2J) + (-2J) = -8J \text{ with degeneracy = 1}
\end{align*}


\section{References}




\end{document}
